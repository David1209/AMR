\documentclass[12pt]{article}
\usepackage{graphicx}
\usepackage{xcolor}
% \usepackage[margin=1.7cm]{geometry}
\usepackage{colortbl}
\usepackage{tikz}
\usepackage{amsmath}
\usepackage{caption}
\usepackage{subcaption}
\usepackage{textcomp}
\newcommand{\todo}[2]{{\color{red}\textbf{Todo voor #1:} #2}}
\begin{document}
\begin{titlepage}
\begin{center}
    \includegraphics[width=\textwidth]{./logo.png}
    \\ [2.5cm]
    \textsc{\Large Autonomous Mobile Robots}
    \\ [0.5cm]
    \textsc{\large Fourth assignment}
    \\ [1cm]
    \hrule
    \vspace{0.3cm}
    \textsc{Particle Filter Based Simultaneous Localization And Mapping}
    \\ [0.3cm]
    \hrule
    \vfill
    \textsc{Ruben Janssen, 10252657 \\ David van Erkelens, 10264019 \\[0.7cm] Department of Computer Science \\ University of Amsterdam \\[0.3cm] \today \\[0.5cm] All code published on}
    \\
    \verb|www.github.com/David1209/AMR/assignment4|
\end{center}
\end{titlepage}
\tableofcontents
\clearpage
\section{Introduction}
In order for an autonomous mobile robot to be able to navigate without a priori knowlegde about the map, it has to determine its position whilst building the map. This can be accomplished by using the Simultaneous Localization and Mapping (SLAM) algorithm. One of the variations of SLAM is FastSLAM, a particle filter based algorithm which has a better performance in real time map building. FastSLAM scales logarithmically with the number of landmarks in the map, and is therefore much more suitable for real-time applications.

\section{Setup}
A NXT robot has been equipped with an omnidirectional camera. Using this NXT, a dataset of images has been made. This is a contraint of the equipment provided, since there was no NXT available to perform real-time experiments. A basic environment has been taped out, and from the point of view of the robot some snapshots have been made. 
\todo{David}{Invoegen van afbeelding van de map}  \todo{Ruben}{Iets vertellen over meegeleverde logfiles}
\section{FastSLAM}
\subsection{Line extraction}
\subsection{Parameters}
\section{Experiments}
\textbf{INSERT PART 2 HERE}
\section{Results}

\section{Conclusion}

\end {document}